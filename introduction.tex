\chapter*{Введение}
\addcontentsline{toc}{chapter}{Введение}

На данный момент генетический алгоритм (ГА) является одним из наиболее исследуемых и развивающихся алгоритмов глобальной оптимизации. К сожалению, до сих пор, по крайней мере, в российской литературе, строго описания алгоритма автором не встречалось. Обычно даны лишь общие рекомендации, которые допускают разночтение в процессе программирования. Либо предложенные схемы применимы лишь для решения тестовых задач. Из-за этого невозможно строго сопоставить друг другу различные исследования по эффективности алгоритмов. Как сказал один из исследователей: «Количество различных эволюционных алгоритмов совпадает с количеством исследователей, работающих в данной области!» \cite{web:makeingsimpleGA}. Данная работа призвана определить некий стандарт генетического алгоритма (далее Стандарт).

Данный документ представляет его версию 3.0 от 3 мая 2013 года.

Последнюю версию документа можно найти по адресу:

\href{https://github.com/Harrix/Standard-Genetic-Algorithm}{https://github.com/Harrix/Standard-Genetic-Algorithm}

\textbf{Цель Стандарта} --- предоставить исследователям единое описание генетического алгоритма для последующей оценки и сравнения эффективности ГА и других алгоритмов. 

Задача, которую решает Стандарт --- определить технологию реализации сГА (стандартный генетический алгоритм). Технология обеспечивает единое понимание и одинаковую реализацию.

В качестве объекта в Стандарте выступает сГА однокритериальной оптимизации на бинарных или вещественных строках.

Стандарт предназначен, в первую очередь, для следующих пользователей:
\begin{itemize}
\item исследователей, занимающиеся  разработкой и исследованиями модификаций ГА (студенты, аспиранты, докторанты, любители-энтузиасты и др.);
\item экспертов, кто может участвовать в оценке результатов работ исследователей.
\end{itemize}


Автор выражает благодарности людям, помогающим в работе данного Стандарта: Бухтоярову Владимиру Викторовичу (\href{mailto:vladber@list.ru}{vladber@list.ru}), Галушину Павлу Викторовичу (\href{mailto:galushin\_pavel@mail.ru}{galushin\_pavel@mail.ru}), Семенкину Евгению Станиславовичу (\href{mailto:eugenesemenkin@yandex.ru}{eugenesemenkin@yandex.ru}), Сергиенко Роману Борисовичу (\href{mailto:romaserg@list.ru}{romaserg@list.ru}), Сопову Евгению Александровичу (\href{mailto:es\_gt@mail.ru}{es\_gt@mail.ru}), Сопову Сергею Александровичу (\href{mailto:sopov.sa@gmail.com}{sopov.sa@gmail.com}).

В данной версии документа по сравнению с предыдущем произведен перевод на \LaTeX\ систему и платформу GitHub, рисунки все переведены в векторный формат, исправлены ошибки и добавлены некоторые замечания.


\clearpage