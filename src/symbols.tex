%%% HarrixLaTeXSymbols
%%% Версия 1.1
%%% Глава о условных обозначениях, которые используются в LaTeX документах. Использую у себя в крупных работах с математическим уклоном. Для использования в связке с проектом HarrixLaTeXDocumentTemplate (https://github.com/Harrix/HarrixLaTeXDocumentTemplate).
%%% https://github.com/Harrix/HarrixLaTeXSymbols
%%% Шаблон распространяется по лицензии Apache License, Version 2.0.

\chapter*{Условные обозначения}
\addcontentsline{toc}{chapter}{Условные обозначения}
\setstretch{0.95}
$a \in A$ --- элемент $ a $ принадлежит множеству $ A $.

$ \bar{x} $ --- обозначение вектора.

$ \arg{f(x)} $ --- возвращает аргумент $x$, при котором функция принимает значение $ f(x) $.

$ Random(X) $ --- случайный выбор элемента из множества $ X $ с равной вероятностью.

$ Random\left ( \left \{x^i \mid p^i \right \} \right ) $ --- случайный выбор элемента $ x^i $ из множества $ X $, при условии, что каждый элемент $ x^i\in X $ имеет вероятность выбора равную $ p^i $, то есть это обозначение равнозначно предыдущему.

$ random(a,b) $ --- случайное действительное число из интервала $ [a; b] $.

$ int(a) $ --- целая часть действительного числа $ a $.

$ \mu(X) $ --- мощность множества $ X $.

\textbf{Замечание.} Оператор присваивания обозначается через знак «$ = $», так же как и знак равенства.

\textbf{Замечание.} Индексация всех массивов в документе начинается с $ 1 $. Это стоит помнить при реализации алгоритма на C-подобных языках программирования, где индексация начинается с нуля.

\textbf{Замечание.} Вызывание трех функций: $ Random(X) $, $ Random\left ( \left \{x_i \mid p_i \right \} \right ) $, $ random(a,b) $ – происходит каждый раз, когда по ходу выполнения формул, они встречаются. Если формула итерационная, то нельзя перед ее вызовом один раз определить, например, $ random(a,b) $ как константу и потом её использовать на протяжении всех итераций неизменной.

\textbf{Замечание.} Надстрочный индекс может обозначать как возведение в степень, так и индекс элемента. Конкретное обозначение определяется в контексте текста, в котором используется формула с надстрочным индексом. 

\textbf{Замечание.} Если у нас имеется множество векторов, то подстрочный индекс обозначает номер компоненты конкретного вектора, а надстрочный индекс обозначает номер вектора во множестве, например, $ \bar{x}^i \in X $ ($i=\overline{1,N}$), $ \bar{x}^i_j \in \left\lbrace 0; 1\right\rbrace  $, ($j=\overline{1,n}$). В случае, если вектор имеет свое обозначение в виде подстрочной надписи, то компоненты вектора проставляются за скобками, например, $ \left( \bar{x}_{max}\right)_j=0$ ($j=\overline{1,n}$). 

\textbf{Замечание.} При выводе матриц и векторов элементы могут разделяться как пробелом, так и точкой с запятой, то есть обе записи $ {\left(\begin{array}{cccccccc}
 1&1&1&1&1&1&1&1
\end{array} \right)}^\mathrm{T} $ и $ {\left(1;1;1;1;1;1;1;1;1 \right)}^\mathrm{T} $ допустимы.

\textbf{Замечание.} При выводе множеств элементы разделяются только точкой с запятой, то есть допустима только такая запись: $ {\left\lbrace 1;1;1;1;1;1;1;1;1 \right\rbrace }^\mathrm{T} $.

\singlespacing
\clearpage