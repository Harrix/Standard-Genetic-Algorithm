\chapter{Вектор параметров генетического алгоритма}\label{StandardGA:section_ParametersGA}

Как мы помним вектор параметров сГА состоит из двух частей:
\begin{equation}
\label{StandardGA:eq:Parameters}
Parameters =\left( \begin{array}{c}ParametersOfBinaryGA\\ParametersOfConvertingIntoBinaryGA
\end{array}\right) .
\end{equation}

Распишем два этих компонента на основании предыдущего материала.

Вначале рассмотрим $ ParametersOfBinaryGA $ (формула \ref{StandardGA:eq:ParametersOfBinaryGA}), компоненты которого определяют параметры стандартного генетического алгоритма на бинарных строках:

\begin{equation*}
\label{StandardGA:eq:ParametersOfBinaryGA2}
ParametersOfBinaryGA=  \left( \begin{array}{c} CountOfFitness \\ TypeOfSel \\ T \\ TypeOfCros \\ TypeOfMutation\\TypeOfForm \end{array}\right).
\end{equation*}

Полагая, что в стандартном генетическом алгоритме при турнирной селекции размер турнира равен $ T=2 $, то можно сократить вектор:

\begin{equation}
\label{StandardGA:eq:ParametersOfBinaryGA3}
ParametersOfBinaryGA=  \left( \begin{array}{c} CountOfFitness \\ TypeOfSel \\ TypeOfCros \\ TypeOfMutation\\TypeOfForm \end{array}\right).
\end{equation}

где 

$ CountOfFitness \in \mathbb{N}$,

$ TypeOfSel \in \left\lbrace ProportionalSelection; RankSelection; TournamentSelection\right\rbrace  $,

$ TypeOfCros \in \left\lbrace SinglepointCrossover; TwopointCrossover; UniformCrossover\right\rbrace  $,

$ TypeOfMutation \in \left\lbrace Weak; Average; Strong\right\rbrace  $,

$ TypeOfForm \in \left\lbrace \begin{array}{c} OnlyOffspringGenerationForming \\  OnlyOffspringWithBestGenerationForming
\end{array}\right\rbrace  $.

Так как $ \mu\left( TypeOfSel\right) =3 $, $ \mu\left( TypeOfCros\right)=3 $, $ \mu\left( TypeOfMutation\right)=3 $, $ \mu\left( TypeOfForm\right)=2 $, то при фиксированных $ CountOfFitness $ и $ N $ (что обычно делают при исследованиях генетического алгоритма) число различных вариантов настроек генетического алгоритма равно:
\begin{eqnarray}
\label{StandardGA:eq:CountOfSettings}
CountOfSettings&=& \mu\left( TypeOfSel\right)\cdot \mu\left( TypeOfCros\right) \cdot\mu\left( TypeOfMutation\right)\cdot\\ & & \cdot\mu\left( TypeOfForm\right)=54.\nonumber
\end{eqnarray}

Это может быть использовано для определения эффективных настроек генетического алгоритма.

Теперь рассмотрим вектор $ ParametersOfConvertingIntoBinaryGA $, определяющий преобразование задачи вещественной оптимизации в задачу бинарной оптимизации.

\begin{equation}
\label{StandardGA:eq:ParametersOfConvertingIntoBinaryGA2}
ParametersOfConvertingIntoBinaryGA=  \left( \begin{array}{c} TypeOfConverting  \\ NumberOfParts_i \end{array}\right),
\end{equation}

где

$TypeOfConverting \in \left\lbrace IntConverting; GrayСodeConverting\right\rbrace $,

$NumberOfParts_i \in \left\lbrace 1;2;\ldots\right\rbrace $, ($i=\overline{1,n}$).

Соотвественно, для стандартного генетического алгоритма на вещественных строк число различных вариантов настроек генетического алгоритма равно:
\begin{eqnarray}
\label{StandardGA:eq:CountOfSettingsReal}
CountOfSettings&=& 54\cdot \mu\left( ParametersOfConvertingIntoBinaryGA\right)=108.
\end{eqnarray}

\textbf{Рекомендации.} Для некоторых параметров существуют рекомендации по выбору значений. Если выбирать не из предложенных вариантов, то алгоритм будет работать, но заданные величины не будут соответствовать реальным, которые используются внутри алгоритма (будут пересчитаны).

\begin{equation}
\label{StandardGA:eq:CountOfFitness2}
CountOfFitness=k_1^2,\text{где } k_1 \in \mathbb{N}, \text{например, } k_1=100.
\end{equation}
\begin{equation}
\label{StandardGA:eq:CountOfFitness}
NumberOfParts_i=2^{\left( k_2\right)_i }-1,\text{где } \left( k_2\right)_i \in \mathbb{N}, \text{например, } \left( k_2\right)_i=14 \left( i=\overline{1,n}\right) .
\end{equation}

Также предлагается способ определения конкретных значений $NumberOfParts_i $, а точнее $ \left( k_2\right)_i $. Для задач вещественной оптимизации в качестве одного из параметра выступает задаваемая точность вычислений $ \varepsilon $. Даная величина несет в себе следующий смысл: запуск генетического алгоритма считается удачным, если найденное алгоритмом решение $ \bar{x}_{submax}$ удовлетворяет условию:
\begin{equation}
\label{StandardGA:eq:varepsilon}
\left| \left( \bar{x}_{submax}\right)_i-\left( \bar{x}_{max}\right)_i\right| \leq\varepsilon, \text{где } i=\overline{1,n}.
\end{equation}

Шаг дискретизации в генетическом алгоритме берется в $ 10 $ раз меньше $ \varepsilon $, чтобы при переходе к задаче бинарной оптимизации ГА мог найти точку рядом с глобальным оптимумом. Итак:
\begin{eqnarray}
\label{StandardGA:eq:StepDis}
h_i&\leq&\frac{\varepsilon}{10};\nonumber\\
\dfrac{Right_i-Left_i}{2^{\left( k_2\right)_i }-1}&\leq&\frac{\varepsilon}{10};\nonumber\\
2^{\left( k_2\right)_i }&\geq&\dfrac{10\left(Right_i-Left_i \right) }{\varepsilon}+1,
\end{eqnarray}

где $ i=\overline{1,n} $, $ \left( k_2\right)_i \in \mathbb{N} $.

Тогда  $ \left( k_2\right)_i $ определяется как минимальное натуральное число, которое удовлетворяет последнему неравенству \ref{StandardGA:eq:StepDis}. На практике выбор решается простым перебором натуральных чисел, начиная с $ 1 $. Это некритично с точки времени выполнения, так как перебор выполняется за всё время работы алгоритма только один раз, да и просмотреть нужно обычно не более пару десятков натуральных чисел начиная с $ 1 $.

Отметим, что мы не можем использовать только одно значение  $ k_2 $, а нужно использовать целый вектор $ \left( k_2\right)_i $ ($ i=\overline{1,n} $). Это объясняется тем, что $ \varepsilon $ для всех вещественных координат одно и то же. Однако границы изменения каждой координаты могут быть различны, а, следовательно, разбиваться интервал каждой координаты при переходе к бинарной задаче оптимизации будет на разное число частей.

В случае если в задаче вещественной оптимизации не приводятся данные об $ \varepsilon $, то выбор $ NumberOfParts_i $ производится пользователем по своему усмотрению.

\textbf{Замечание.} Когда мы рассматривали сГА как некий оператор (3.1), в его входных параметрах нигде не упоминается $ n $ --- длина вектора $ \bar{x} $, $ Left_i $ и $ Right_i $ ($ i=\overline{1,n} $) --- границы изменения компонент $ \bar{x} $, если решается задача вещественной оптимизации. Это связано с тем, что данная информация заключена во входном параметре $ X $ --- множестве всех возможных решений. Но при программировании лучше передавать только эти характеристики множества $ X $, а не само всё множество.

\textbf{Замечание.} При проведении исследований эффективности сГА и других алгоритмов на его основе, в качестве параметров алгоритма удобнее прописывать полные их наименования, а не те буквенные обозначения, которые приняты в данном документе. Это будет удобнее для анализа данных другими исследователями, которые не читали данный документ. Автором используется формат \textbf{Harrix Optimization Testing} (\href{https://github.com/Harrix/HarrixFileFormats}{https://github.com/Harrix/HarrixFileFormats}) для проведения исследований. И в них приняты следующие обозначения:

\begin{eqnarray}
\label{StandardGA:eq:ParametersOfRus}
CountOfFitness& = &\text{\textit{Максимальное допустимое число}}\\
& &\text{\textit{вычислений целевой функции}};\nonumber\\
TypeOfSel& = &\text{\textit{Тип селекции}};\nonumber\\
TypeOfCros& = &\text{\textit{Тип скрещивания}};\nonumber\\
TypeOfMutation& = &\text{\textit{Тип мутации}};\nonumber\\
TypeOfForm& = &\text{\textit{Тип формирования нового поколения}};\nonumber\\
ProportionalSelection& = &\text{\textit{Пропорциональная селекция}};\nonumber\\
RankSelection& = &\text{\textit{Ранговая селекция}};\nonumber\\
TournamentSelection& = &\text{\textit{Турнирная селекция}};\nonumber\\
SinglepointCrossover& = &\text{\textit{Одноточечное скрещивание}};\nonumber\\
TwopointCrossover& = &\text{\textit{Двухточечное скрещивание}};\nonumber\\
UniformCrossover& = &\text{\textit{Равномерное скрещивание}};\nonumber\\
Weak& = &\text{\textit{Слабая мутация}};\nonumber\\
Average& = &\text{\textit{Средняя мутация}};\nonumber\\
Strong& = &\text{\textit{Сильная мутация}};\nonumber\\
OnlyOffspringGenerationForming& = &\text{\textit{Только потомки}};\nonumber\\
OnlyOffspringWithBestGenerationForming& = &\text{\textit{Только потомки и копия лучшего индивида}};\nonumber
\end{eqnarray}


\clearpage