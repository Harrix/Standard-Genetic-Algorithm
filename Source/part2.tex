\chapter{Определения генетического алгоритма}\label{StandardGA:section_def}

Введем некоторые понятия, которыми будем пользоваться в данном Стандарте.

\textbf{Стандартный генетический алгоритм} --- генетический алгоритм, описанный в данном документе. Он считается базовым для исследования эффективности других ГА. Если не оговорено иное, то под генетическим алгоритмом понимается стандартный генетический алгоритм. Для обозначения также используется сокращение сГА.

\textbf{Бинарная строка} --- вектор-столбец $\bar{x}$ конечной длины $ n $, компоненты которого могут принимать значения из множества $ \left\lbrace 0; 1\right\rbrace  $. В принципе не совсем верное название, так как по сути это вектор-столбец, а не вектор-строка.

\textbf{Бинарный вектор} --- то же самое, что и бинарная строка.

\textbf{Вещественная строка} --- вектор-столбец $\bar{x}$ конечной длины $ n $, компоненты которого могут принимать значения из множества $ x_i\in\left\lbrace Left_i;Right_i\right\rbrace (i=\overline{1,n}) $. По своей сути не совсем верное название, так как по сути это вектор-столбец, а не вектор-строка.

\textbf{Вещественный вектор} --- то же самое, что и вещественная строка.

\textbf{Решение} --- любой элемент $\bar{x}$ из множества $ X $.

\textbf{Точка поискового пространства} --- то же самое, что и решение.

\textbf{Возможное решение} --- любое решение $\bar{x}$ из множества $ X $.

\textbf{Допустимое решение} --- решение $\bar{x}$ из множество $ X $, которое удовлетворяет условиям $ g_i\left (\bar{x}\right )\leq 0 \left( i=\overline{1,m_1},\nonumber\right)  $ и $ h_j\left (\bar{x}\right )= 0 \left(  j=\overline{1,m_2}\right)  $.

\textbf{Генотип} --- то же самое, что и бинарная строка.

\textbf{Индивид} --- то же самое, что и бинарная строка. 

\textbf{Хромосома} --- то же самое, что и бинарная строка.

\textbf{Ген} --- компонент $ x_i $ бинарного вектора $\bar{x}\in X$.

\textbf{Бит} --- компонент $ x_i $ бинарного вектора $\bar{x}\in X$.

\textbf{Фенотип} --- некий объект, который был закодирован в бинарную, строку при переходе от задачи оптимизации на произвольном пространстве поиска к задаче оптимизации на бинарных строках. Например, вещественный вектор преобразуется в бинарную строку с помощью кода Грея.

\textbf{Поколение} --- одна итерация работы сГА.

\textbf{Популяция} --- множество индивидов, обрабатываемых на текущем поколении.

\textbf{Целевая функция} --- то же самое, что и функционал $ f\left( \bar{x}\right)  $.

\textbf{Оптимизируемая функция} --- то же самое, что и функционал $ f\left( \bar{x}\right)  $.

\textbf{Функция пригодности} --- преобразованная целевая функция для использования в генетическом алгоритме на бинарных строках.

\textbf{Пригодность} --- значение функции пригодности.

\textbf{Поисковое пространство} --- множество всех возможных векторов $ X $.

\textbf{Множество решений} --- то же самое что и поисковое пространство.

Остальные определения будут вводиться по мере их упоминания в тексте.

\textbf{Замечание.} Понятия «генотип», «хромосома», «индивид» обозначают по своей сути одно и то же. Но в связи с тем, что генетический алгоритм имитирует настоящую эволюцию, то на различных этапах имитации более подходящими являются разные понятия из биологии.

Также стоить отметить, что здесь эти понятия отождествляются с бинарной строкой, как возможным решением, а не любым решением из первоначальной задачи оптимизации (там могут присутствовать также вещественные строки). Это объясняется тем, что в описанном в Стандарте генетическом алгоритме на уровне операторов используются именно бинарные строки. А задача вещественной оптимизации решается путем сведения к задаче бинарной оптимизации путем дискретизации поискового пространства, операторов кодирования и декодирования. Но это не означает, что в других вариантах генетического алгоритма, не описанных в Стандарте, понятия «генотип», «хромосома», «индивид» будут отождествляться с бинарными строками.

\clearpage